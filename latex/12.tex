\documentclass{report}

\title{The Impact of Social Media on Society}
\author{Author Name}
\date{\today}

\begin{document}

\maketitle

\tableofcontents

\chapter{Introduction}
Social media has revolutionized the way people communicate, share information, and connect with each other. This report explores the impact of social media on various aspects of society, including communication, mental health, and information dissemination.

\chapter{Methods}
This study employs a mixed-methods approach to understand the impact of social media. Quantitative data was collected through surveys distributed to 500 participants, while qualitative data was gathered through in-depth interviews with 30 individuals. Data analysis involved statistical methods to interpret the survey results and thematic analysis for the interview data.

\chapter{Results}
The survey results indicated that 75% of participants use social media daily, with the majority citing it as a primary source of news and information. However, 60% reported experiencing negative effects on their mental health due to social media use. Interviews revealed that social media influences communication patterns, with participants noting both positive and negative impacts on their relationships.

\chapter{Discussion}
The results suggest that while social media offers significant benefits in terms of connectivity and information access, it also poses challenges to mental health and interpersonal communication. The discussion addresses the dual nature of social media's impact, comparing the findings with existing literature and highlighting potential areas for intervention and policy development.

\chapter{Conclusion}
In conclusion, social media has a profound impact on society, with both positive and negative effects. It enhances connectivity and access to information but also contributes to mental health challenges. Future research should focus on developing strategies to mitigate the negative impacts while maximizing the benefits of social media.

\end{document}
